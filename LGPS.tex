\documentclass[twoside,twocolumn]{article}
%\documentclass[twoside]{article}
\date{}
\usepackage[sc]{mathpazo}
\usepackage[T1]{fontenc}
\linespread{1.05}
\usepackage{multirow}
\usepackage{microtype}
\usepackage[english]{babel}
\usepackage[hmarginratio=1:1,top=15mm, bottom = 15mm, left = 10mm, right = 10mm, columnsep=15pt]{geometry}
\usepackage[hang,small,labelfont=bf,up,textfont=up]{caption}
\usepackage{booktabs}
\usepackage{lettrine}
\usepackage{enumitem}
\usepackage{gensymb}
\usepackage{amsmath}
\usepackage{amsfonts}
\setlist[itemize]{noitemsep}
\usepackage{titlesec}
\renewcommand\thesection{\Roman{section}}
\renewcommand\thesubsection{\roman{subsection}}
\titleformat{\section}[block]{\large\scshape\centering}{\thesection.}{1em}{}
\titleformat{\subsection}[block]{\large}{\thesubsection.}{1em}{}
\usepackage{fancyhdr}
%\pagestyle{fancy}
\usepackage{titling}
\usepackage{hyperref}
\newcommand{\tabincell}[2]{\begin{tabular}{@{}#1@{}}#2\end{tabular}}
%----------------------------------------------------------------------------------------
\usepackage{graphicx}
\setlength{\droptitle}{-4\baselineskip}
%\pretitle{\begin{center}\LARGE\bfseries}
\pretitle{\begin{center}\LARGE}
\posttitle{\end{center}}
\title{Balancing the Tradeoff Between Accuracy and Efficiency\\ for Saliency Prediction}
\author{%
%\textsc{Shanghua Xiao} \\
%\normalsize Sichuan University
}
\begin{document}
\maketitle
\begin{abstract}\
1) Saliency prediction draws attention, and many application are developed. But the improvement on efficiency is rarely investigated. 2) In this work, we explore how to balance the accuracy and efficiency for saliency prediction tasks. 3) To perform accurate saliency prediction, object and semantic level information are involved by multi-scale inputs. And to achieve efficiency, we reduce the spatial complexity from two perspectives: parameter number and intermediate feature map sizes. We use truncated pretrained model to scale down the parameter number. And we apply regional input to scale down the intermediate feature map sizes. 4) We evaluate the accuracy of our network on MIT300 and CAT2000 benchmark dataset and achieve promising results comparing with state-of-the-art methods. Also we evaluate the efficiency on memory cost for producing one saliency map of our network, and outperform other recent proposals by a considerable margin.

\end{abstract}

\section{Introduction}

\par 1) Saliency prediction aims to get a mask map and stuff...introduce. 2) Saliency prediction draws attention, and many application are developed. 2) Many saliency prediction application lie in robotics, where real-time is usually a requirement, and computational resources are scarce. Given that saliency prediction in robotics has been extensively used to speed-up other visual tasks for the robots such as \cite{borji2010online,dankers2007reactive}. So it is equally important for a saliency prediction approach to perform efficiently as well as accurately.

\par 1) To boost the performance in accuracy, many recent methods apply large-scale image classification models (\cite{Simonyan14c,szegedy2015going,krizhevsky2012imagenet}) for pretraining. 2) While achieving significant improvement on accuracy, the requirement on computational resources may limit their scope of application for the perspective of preprocessing method. 3) The limitation is caused by the paradox between low computational demands for saliency prediction and high computational consumption for classification models. 4) According to feature integration theory, saliency prediction only need to assign a proper salient activation for features rather than classes. For example, "Eskimo dog" and "German shepherd" are two different classes in classification, and need to be represented in two different feature activation in order to perform accurate classification. While in saliency prediction, these two classes are just two minor modification for "dog" and can use one shared feature for salient activation.

\par 1) In this work, we propose a novel approach with the effort of balancing the tradeoff between accuracy and efficiency for saliency prediction. 2) To achieve accurate saliency prediction, multiscale regional input is applied to involve object and semantic level stimuli, and two independent fully convolutional neural networks are applied to preserve the object and semantic level information simultaneously. And the efficiency is achieved via two ways: regional input for scaling down the data blob, and truncated pretrained model for reducing the parameter number and thus model complexity.

\par 1) We evaluate the performance of our network on multiple benchmarks with held out test set, and achieve promising results comparing with state-of-the-art methods. 2) We also evaluate the computational complexity with average memory cost for producing on saliency map on an unified environment, and give a result with 14-26 times less than other recent methods \cite{}.

\par We summarise the contribution of our work as follows:
\begin{itemize}
\item We propose a novel approach for capturing object and semantic level information simultaneously for saliency prediction, and achieve promising results in multiple metrics.
\item By scaling down the parameter number and intermediate feature map sizes, we reduce the computational complexity for producing one saliency map by a considerable margin comparing with other recent methods.
\end{itemize}  

\section{Related Works}

\par 1) In our proposed approach, we mainly investigate how to achieve accurate saliency prediction by capturing both object and semantic level information, and how to perform efficient saliency prediction by reducing the memory cost. In this section, we review some related works in these two direction.

\par 1) In \cite{itti2001computational}, Itti \emph{et al.} shows that object and semantic level information are both vital for saliency prediction. Following this, many methods give their solution for capturing multilevel features from different directions. Deep Fixation \cite{kruthiventi2015deepfix} propose a FCNN based approach to unify the multiscale feature learning. In their work, "inception module" is utilized to extract complex semantic features, and "kernels with holes" are used to capture global context via large receptive field. In the SALICON proposal \cite{jiang2015salicon}, a fine resolution and a coarse resolution (with halved scale) image are send to two kernel-sharing networks to learn feature from multiple scales. Deep Gaze \cite{kummerer2014deep} propose another way to learn multilevel information by combining the feature maps from a selection of layers. In their work, five layers at different levels are rescaled and cropped to form a 3-dimensional tensor, then a final saliency map is extracted through a four-layer-network with kernel size of $1\times1$. Another noticeable work is Multi-resolution Convolutional Neural Network (Mr-CNN). In the Mr-CNN proposal, three image slices with size of $42\times42$ are cropped from three resized input image ($150\times150, 250\times250, 400\times400$) as inputs. The slices are feed into three kernel-sharing CNNs to learn features from different resolutions, then to two fully connected layers to perform salient classification.

\par 1) Jiang et al. and \cite{cbmmsaliency}, efficiency. 2) Shallow and Deep \cite{pan2016shallow}, explore memory cost: parameter number and data blob. Even with small parameter number, memory cost of data blob can be significantly big. 

\par 1) Our work mainly focus on how to get accurate saliency prediction while maintaining the efficiency.

\section{Proposed Approach}\

\subsection{Network Architecture}

\subsection{Blending Method}

\subsection{Refinements on Saliency Map}

\section{Experiments}

\subsection{Datasets}

\par 1) We train our network on SALIency in CONtext (SALICON) dataset. \cite{jiang2015salicon}, which provides a rich number of rich semantical information image samples, with 10,000 training samples, 5,000 validation samples and 5,000 testing samples from multiple categories. 

\par 1) At validation and testing stage, we conduct the evaluation on three widely used dataset, namely, MIT1003, CAT2000 and MIT300 dataset. 2) The MIT1003 dataset contains 1003 . 2) The CAT2000 dataset is consist of one training set with accessable ground truth and one testing set with held out ground truth fixation maps. 3) The MIT300 

\subsection{Evaluation Metrics}

\par 1) During evaluation, multiple metrics are used, since previous study by Riche et al. \cite{riche2013saliency} shows that no single metric has concrete guarantee of fair comparison. 

\paragraph{Area Under Roc Curve (AUC) and Normalized Saliency Scanpath (NSS):}

\paragraph{Similarity (Sim), Correlated Coefficient (CC), Earth Mover's Distance (EMD) and Kullback-Leibler Divergence (KL):}

\subsection{Implementation Details}

\subsection{Results}

\section{Discussion and Conclusion}\

\par 1) Though the application of adaptive central bias improved the performance of our network on multiple dataset, the usage on prior biases need for further discussion. Prior biases, such as central bias, are widely applied in saliency prediction proposals, and remarkably improved the performance. Also, central bias in ground truth fixation maps is a common phenomenon in multiple saliency dataset such as MIT1003 and CAT2000. 2) However, we find that there is a worth-noticing similarity between overwhelming central biasing and Autism Spectrum Disorder (ASD) \cite{wang2015}, which is an attention disorder disease that has been proven will cause serious central bias for human when free viewing. See Figure.4, fixation maps from patients (6 and 8) with ASD overly focus their attention on the center of the stimuli while ignoring the actual salient object such as elephant and desk lamp.

\par While in many saliency prediction tasks, especially the task in real-time interactive environment such as automatic driving and robotic controlling, predicted fixation should be shifting rapidly among different potential attention attractive spots. Thus we argue that overly relying on prior bias maybe a misleading direction for improving saliency prediction performance.

\par In this work, we present a novel approach for saliency prediction with the effort of balancing the tradeoff between accuracy and efficiency. The proposed method achieve promising results on multiple benchmark datasets over different metrics comparing with state-of-the-art methods, while reducing the runtime and static memory cost of the network by a considerable margin.

\bibliographystyle{plain}
\bibliography{GLPS}
\end{document}
